\documentclass[12pt, letterpaper]{article}
\usepackage[utf8]{inputenc}

\usepackage{graphicx}
\graphicspath{ {images/} }

\usepackage{amsmath}

\title{First document}
\author{Michael Fedell \thanks{sourced from overleaf.com}}
\date{January 2019}

\begin{document}

\maketitle
\tableofcontents

\newpage

\begin{abstract}
    This is a simple paragraph at the beginning of the
    document. A brief introduction about the main subject.
\end{abstract}

\section{Basics}

We now have a title prepared with author and date, as well as some basic section and subsection labeling.
\\
This \LaTeX{} document is starting to look great!

\subsection{Styling}
Some of the \textbf{greatest}
discoveries in \underline{science}
were made by \textbf{\textit{accident}}.

\vspace{10mm}

In case you didn't catch that,
\medskip\\
Some of the greatest \emph{discoveries}
in science
were made by accident.
\\
\textit{Some of the greatest \emph{discoveries}
in science
were made by accident.}
\\
\textbf{Some of the greatest \emph{discoveries}
in science
were made by accident.}

\subsection{Graphics}
The universe is immeense and it seems to be homogeneous, in a large scale, everywhere we look.

\smallskip
\includegraphics{moon}

There's a picture of the moon above

\begin{figure}[h]
    \centering
    \includegraphics[width=0.25\textwidth]{mesh}
    \caption{a nice plot}
    \label{fig:mesh1}
\end{figure}

As you can see in the figure~\ref{fig:mesh1}, the
function grows near 0. Also, on page~\pageref{fig:mesh1}
is the same example.

\subsection{Lists}

\begin{itemize}
    \item The individual entries are indicated with a black dot, a so-called bullet.
    \item The text in the entries may be of any length.
\end{itemize}

\begin{enumerate}
    \item This is the first entry in our list
    \item The list numbers increase with each entry we add
\end{enumerate}

\subsection{Math}

In physics, the mass-energy equivalence is stated
by the equation $E=mc^2$, discovered in 1905 by Albert Einstein.

The mass-energy equivalence is described by the famous equation

\[E=mc^2\]

discovered in 1905 by Albert Einstein.
In natural units ($c = 1$), the formula expresses the identity

\begin{equation}
E=m
\end{equation}

Subscripts in math mode are written as $a_b$ and superscripts are written as $a^b$. These can be combined an nested to write expressions such as

$$T^{i_1 i_2 \dots i_p}_{j_1 j_2 \dots j_q} = T(x^{i_1},\dots,x^{i_p},e_{j_1},\dots,e_{j_q})$$

We write integrals using $\int$ and fractions using $\frac{a}{b}$. Limits are placed on integrals using superscripts and subscripts:

$$\int_0^1 \frac{1}{e^x} =  \frac{e-1}{e}$$

Lower case Greek letters are written as $\omega$ $\delta$ etc. while upper case Greek letters are written as $\Omega$ $\Delta$.

Mathematical operators are prefixed with a backslash as $\sin(\beta)$, $\cos(\alpha)$, $\log(x)$ etc.

\section{Document Structure and Formatting}

This is the first section.

Lorem  ipsum  dolor  sit  amet,  consectetuer  adipiscing
elit.   Etiam  lobortisfacilisis sem.  Nullam nec mi et
neque pharetra sollicitudin.  Praesent imperdietmi nec ante.
Donec ullamcorper, felis non sodales...

\section{Second Section}

Lorem ipsum dolor sit amet, consectetuer adipiscing elit.
Etiam lobortis facilisissem.  Nullam nec mi et neque pharetra
sollicitudin.  Praesent imperdiet mi necante...

\subsection{First Subsection}
Praesent imperdietmi nec ante. Donec ullamcorper, felis non sodales...

%This will force the unnumbered section to appear in the table of contents%
\addcontentsline{toc}{section}{Unnumbered Section}
\section*{Unnumbered Section}
Lorem ipsum dolor sit amet, consectetuer adipiscing elit.
Etiam lobortis facilisissem

\section{Tables}
Table \ref{table:data} is an example of referenced \LaTeX{} elements.

\begin{table}[h!]
    \centering
    \begin{tabular}{||c c c c||}
        \hline
        Col1 & Col2 & Col2 & Col3 \\ [0.5ex]
        \hline\hline
        1 & 6 & 87837 & 787 \\
        2 & 7 & 78 & 5415 \\
        3 & 545 & 778 & 7507 \\
        4 & 545 & 18744 & 7560 \\
        5 & 88 & 788 & 6344 \\ [1ex]
        \hline
    \end{tabular}
    \caption{Table to test captions and labels}
    \label{table:data}
\end{table}

\end{document}