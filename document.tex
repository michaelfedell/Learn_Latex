\documentclass[12pt, letterpaper]{article}
\usepackage[utf8]{inputenc}
\usepackage{graphicx}
\graphicspath{ {images/} }

\title{First document}
\author{Michael Fedell \thanks{sourced from overleaf.com}}
\date{January 2019}

\begin{document}
\maketitle
We now have a title prepared with author and date.
This \LaTeX{} document is starting to look great!

\newpage

Some of the \textbf{greatest}
discoveries in \underline{science}
were made by \textbf{\textit{accident}}.


In case you didn't catch that,

Some of the greatest \emph{discoveries}
in science
were made by accident.

\textit{Some of the greatest \emph{discoveries}
in science
were made by accident.}

\textbf{Some of the greatest \emph{discoveries}
in science
were made by accident.}

\newpage

The universe is immeense and it seems to be homogeneous, in a large scale, everywhere we look.

\includegraphics{moon}

There's a picture of the moon above

\newpage

\begin{figure}[h]
    \centering
    \includegraphics[width=0.25\textwidth]{mesh}
    \caption{a nice plot}
    \label{fig:mesh1}
\end{figure}

As you can see in the figure~\ref{fig:mesh1}, the
function grows near 0. Also, in the page~\pageref{fig:mesh1}
is the same example.
\end{document}
